%Comment out lines 3-4 if you want slides, line 2 if you want notes
%\documentclass[ignorenonframetext, red]{beamer}
\documentclass[11pt,a4paper]{article}
\usepackage{beamerarticle}
\mode<article>
{
	\usepackage[colorlinks,
	pdfauthor="Len Thomas",
	pdftitle="MT4113 Lecture notes"]{hyperref}
    \usepackage[a4paper,margin=1cm,footskip=.5cm]{geometry}
}
\mode<presentation>
{
%    \usetheme{Singapore}
  \pgfdeclareimage[height=1cm]{StA-logo}{logo}
  \logo{\pgfuseimage{StA-logo}}
    \usetheme{EastLansing}
%    \usetheme{JuanLesPins}
%    	\usecolortheme{sidebartab}
	\setbeamertemplate{navigation symbols}{\insertsectionnavigationsymbol}
}

\usepackage[english]{babel}
\usepackage[latin1]{inputenc}
\usepackage[T1]{fontenc}
\usepackage{epsf,graphics,graphicx,fancyhdr,color,amsmath,url,enumerate,alltt}
\usepackage{ifthen}


%Commented out proportional font -- use a monospaced font instead
% -- better for displaying the tables with binary numbers in them, etc, in this lecture
%\usepackage[sfdefault]{cabin}
\usepackage[scaled=0.85]{beramono}


\newcommand{\bc}{\begin{center}}
\newcommand{\ec}{\end{center}}
\newcommand{\bn}{\begin{enumerate}}
\newcommand{\en}{\end{enumerate}}
\newcommand{\be}{\begin{eqnarray}}
\newcommand{\ee}{\end{eqnarray}}
\newcommand{\bes}{\begin{eqnarray*}}
\newcommand{\ees}{\end{eqnarray*}}

\title[MT4113]{MT 4113, Computing in Statistics}

\subtitle{Lecture 4 - An introduction to computer arithmetic}

\author[Lecture 4]{Len Thomas}

%\institute{School of Mathematics and Statistics, University of St Andrews}

\date[1/10/2018] {Oct 1 2018}
\begin{document}

\begin{frame}
  \titlepage
\end{frame}

\mode<article>{
\maketitle
}

\mode<presentation> {
\begin{frame}
  \frametitle{Outline}
	\tableofcontents
\end{frame}
}

%-----------------------------------------------------------------

\section{An Introduction to Computer Arithmetic}

\subsection{Introduction}

\begin{frame}
	\frametitle{About this lecture}

This is a rough guide to how computers store and manipulate numbers\footnote{A much more
in-depth treatment is in, e.g., Chapter 2 of Gentle, J.E. Computational Statistics, available from the library as an electronic book.}
\end{frame}

\begin{frame}[fragile]
	\frametitle{Motivating example using \texttt{R}}

What does this code do?

\begin{verbatim}
x <- 0.1
x <- x + 0.05
x

if (x == 0.15) {
  cat("Len is a great guy!\n") 
} else {
  cat("Len is a loser!\n")
}
\end{verbatim}

\end{frame}

\begin{frame}
	\frametitle{Numbers on computers}

	\begin{itemize}
		\item Computers really can't do arithmetic.  They recognize two states: on (1) and off (0).  Numbers on computers are really a collection of binary digits (`bits') -- they only have meaning by human convention.
		
		\item Numbers are represented on computers in two ways:
		\begin{itemize}
      \item fixed point (integers)
	    \item floating point (reals)
		\end{itemize}
	\end{itemize}

\end{frame}

\begin{frame}
	\frametitle{Definitions: Data size}

\begin{itemize}
    \item bit: short for \underline{bi}nary digi\underline{t}
      \\smallest unit of storage on a machine -- holds either 0 or 1
    \item byte: short for \underline{by}nary \underline{te}rm \footnote{Computer scientists can't spell!}
      \\8 bits -- enough to hold a single character of text (see later)
    \item word: natural data size of a computer.  Depends on the CPU -- for most desktop systems currently it is 64 bits (or 32 bits for older machines).
    \item kilobyte (K): $2^{10}=1,024$ bytes
    \item megabyte (MB): $2^{20}=1,048,576$ bytes$ = 1,024$ K
    \item gigabyte (GB): $2^{30}=1,073,741,824$ bytes$\footnote{Hard disk drive manufacturers sometimes prefer to use $1,000,000,000$ bytes as it makes their drives look bigger.} = 1,024$ MB
\end{itemize}

\end{frame}

\subsection{Fixed point}[fragile]

\begin{frame}
	\frametitle{Fixed point representations}

	\begin{itemize}
	    \item Example representation: signed integer\footnote{there are other more efficient representations, e.g., 1s complement and 2s complement}
	    \begin{itemize}
	        \item 1st bit is the sign (0=+ve, 1=-ve)
	        \item Remaining bits are the absolute value in base 2
	    \end{itemize}
	    \item <2-> E.g., on a 4-bit system:
	    \bc
	    \begin{tabular}{ll}
	        0000 is 0 & 1000 is -0 \\
	        0001 is 1 & 1001 is -1 \\
	        0010 is 2 & 1010 is -2 \\
	        0011 is 3 & 1011 is -3 \\
	        $\vdots$  & $\vdots$\\
	        0111 is 7 & 1111 is -7
	    \end{tabular}
	    \ec
	\end{itemize}
\end{frame}

\begin{frame}[fragile]
	\frametitle{Fixed point arithmetic - addition}

	\begin{itemize}
	    \item E.g. addition:
	    \bc
	    \begin{tabular}{rrl}
	        3  & 0011 &\\
	        +2 & 0010 &\\
	        \cline{1-2}
	        5  & 0101 & add
	    \end{tabular}
	    \ec
	\end{itemize}

\end{frame}

\begin{frame}[fragile]
	\frametitle{Fixed point arithmetic - multiplication}

	\begin{itemize}
	    \item Multiply by a power of 2 - just shift the bits left and fill in rightmost bits with zeros
	    \bc
	    \begin{tabular}{rrl}
	        3  & 0011 &\\
	        x2 & 0010 &\\
	        \cline{1-2}
	        6  & 0110 & x2 $\Rightarrow$ shift left one
	    \end{tabular}
	    \ec
	    \item <2-> If not a power of 2, use the `shift and add' algorithm: express as a sum of powers of 2.\\
			E.g., $2\times3 = (2\times2) + (2\times1)$
	    \item <3-> \bc
	    \begin{tabular}{rrl}
	        2  & 0010 &\\
	        x3 & 0011 &\\
	        \cline{1-2}
					4 & 0100 & x2 $\Rightarrow$ shift left one\\
					2 & 0010 & x1 $\Rightarrow$ shift left none\\
	        \cline{1-2}
	        6  & 0110 & add
	    \end{tabular}
	    \ec
	\end{itemize}

\end{frame}

\begin{frame} [fragile]
	\frametitle{Fixed point arithmetic - subtraction and division}

	\begin{itemize}
			\item <1-> Dividing by a power of two is easy 
			\begin{itemize}
				\item <2->just shift right (and drop any fractional part)
			\end{itemize}
	    \item <3-> Otherwise, dividing and subtracting are not very intuitive (or fast) in the signed integer representation, so we do not cover them explicitly here
			\item <3-> They are better (easier) in other representations -- e.g., 1s or 2s complement
	\end{itemize}

\end{frame}

\begin{frame}[fragile]
	\frametitle{Fixed point overflow}

    \bc
    \begin{tabular}{rr}
        3  & 0011\\
        +5 & 0101\\
        \cline{1-2}
        0  & 0000
    \end{tabular}
    \ec
\begin{itemize}
    \item <2->Machine needs a protocol to deal with and warn user of fixed point overflow
\end{itemize}
\end{frame}

\begin{frame}[fragile]
	\frametitle{Fixed point numbers in R}

\begin{itemize}
    \item <1->In R, the fixed point data type is called \texttt{integer}.
		\item <2->Integers aren't actually used much in R, except in loops and indexing elements in arrays.
		\item <3->Integers in R are stored in 32 bits.
\begin{semiverbatim}\mode<presentation>{\tiny}{}
> as.integer(2 ^ 31 - 1)
[1] 2147483647
> as.integer(2 ^ 31 - 1) + as.integer(1)
[1] NA
Warning message:
In as.integer(2 ^ 31 - 1) + as.integer(1) : NAs produced by integer overflow\end{semiverbatim}
\end{itemize}
\end{frame}

\begin{frame}[fragile]
	\frametitle{Summary: Fixed point representations}

	 \begin{itemize}
	    \item Advantages:
	    \begin{itemize}
	        \item Results are exact
	        \item Fast (compared with floating point)
	    \end{itemize}
	    \item <2-> Disadvantages:
	    \begin{itemize}
	        \item Limited applicability: integers and related (e.g., financial applications, dates\footnote{Dates can use unsigned integers representations})
	        \item Limited range
	    \end{itemize}
	\end{itemize}

\end{frame}

\subsection{Floating point}

\begin{frame}
	\frametitle{Floating point representations}

	\begin{itemize}
	    \item Example $S .F \times B^E$ e.g., $+.1 \times 10^{+1}$
	    \begin{itemize}
	        \item S = sign
	        \item F = fraction -- unsigned integer.  Typically `normalized' so that 1st digit is non-zero.
	        \item B = base
	        \item E = exponent -- signed integer.
	    \end{itemize}
	    \item On a computer
	    \begin{itemize}
	        \item B is 2
	        \item S is 1 bit (0=+, 1=-)
	        \item F and E are allocated a fixed number of bits each
	    \end{itemize}
	\end{itemize}
\end{frame}
\begin{frame}[fragile]
	\mode<presentation>{\frametitle{Floating point representations}}
	\begin{itemize}
	    \item For example, consider an 8-bit floating point representation
	    with
	    \begin{itemize}
	        \item 1 bit for the sign
	        \item 3 for the exponent (a signed integer)
	        \item 4 for the fraction (an unsigned integer)
	    \end{itemize}
	\end{itemize}
	    \bc
	    {\tiny
	    \begin{tabular}{rcrcrcrrr}
	        & & \multicolumn{1}{c}{decimal} &   & \multicolumn{1}{c}{binary} & &
	            \multicolumn{1}{c}{sign} & \multicolumn{1}{c}{exponent} &
	            \multicolumn{1}{c}{fraction}\\
	        1 & = & $+.1 \times 10^{+1}$ & = & $+.1_{\mbox{two}} \times 2^{+1}$ & = &
	            0 & 001 & 1000\\
	        3 & = & $+.3 \times 10^{+1}$ & = & $+.11_{\mbox{two}} \times 2^{+2}$ & = &
	            0 & 010 & 1100\\
	    $-1/8$ & = & $-.125 \times 10^{+0}$ & = & $-.1_{\mbox{two}} \times 2^{-2}$ & = &
	            1 & 110 & 1000\\
	    \end{tabular}
	    }
	    \ec
\end{frame}

\begin{frame}[fragile]
	\frametitle{Limitations}

	\begin{itemize}
    \item Given that the exponent and fraction are stored in a fixed number of bits, there are limitations both to the size and accuracy with which numbers can be stored
    \item <2-> Size: in our example
    \begin{itemize}
        \item Largest possible number is 
				\item <3-> $+.1111_{\mbox{two}} \times 2^{3} = 7.5$
        \item <4-> Smallest possible positive number\footnote
        {if we allow the leading digits of the fraction to be zero (`de-normalized')}
        is 
				\item <5-> $+.0001_{\mbox{two}} \times 2^{-3} = 1/128 = 0.0078125$
    \end{itemize}
	\end{itemize}
\end{frame}
\begin{frame}
	\mode<presentation>{\frametitle{Limitations}}
	\begin{itemize}
    \item Accuracy:
    \bc
    \begin{tabular}{rrrrr}
        & \multicolumn{1}{c}{binary} &
            \multicolumn{1}{c}{sign} & \multicolumn{1}{c}{exponent} &
            \multicolumn{1}{c}{fraction}\\
        1 & $.1000_{\mbox{two}} \times 2^{+1}$ &
            0 & 001 & 1000\\
        +1/128 & $+.0001_{\mbox{two}} \times 2^{-3}$ &
            0 & 111 & 0001\\
        \hline
        1 & $.1000_{\mbox{two}} \times 2^{+1}$ &
            0 & 001 & 1000
    \end{tabular}
    \ec
    \begin{itemize}
        \item to see why, re-express 1/128 as
		    \bc
		    \begin{tabular}{rrrrr}
		        +1/128 & $+.00000001_{\mbox{two}} \times 2^{+1}$ &
		            0 & 001 & 00000001\\
		    \end{tabular}
		    \ec
    \end{itemize}
	\end{itemize}
\end{frame}
\begin{frame}[fragile]
	\mode<presentation>{\frametitle{Limitations}}
	\begin{itemize}
    \item Let $\epsilon$ be the smallest positive number, such that $\epsilon + 1.0 \neq 1.0$
    \item In our example
    \begin{itemize}
			\item <2-> $\epsilon = 1/8$:
    \bc
%    \ifthenelse{\talk = 1}{\small}{}
    \begin{tabular}{rrrrr}
        & \multicolumn{1}{c}{binary} &
            \multicolumn{1}{c}{sign} & \multicolumn{1}{c}{exponent} &
            \multicolumn{1}{c}{fraction}\\
        1 & $.1000_{\mbox{two}} \times 2^{+1}$ &
            0 & 001 & 1000\\
        +1/8 & $+.1000_{\mbox{two}} \times 2^{-2}$ &
            0 & 110 & 1000\\
        \hline
        1 1/8 & $.1001_{\mbox{two}} \times 2^{+1}$ &
            0 & 001 & 1001
    \end{tabular}
    \ec
        \item <2-> to see why, re-express 1/8 as
		    \bc
		    \begin{tabular}{rrrrr}
		        +1/8 & $+.0001_{\mbox{two}} \times 2^{+1}$ &
		            0 & 001 & 0001\\
		    \end{tabular}
		    \ec
    \end{itemize}
    \item <3->In general, $\epsilon$, the  `machine epsilon' (or closely related `machine unit'), governs the relative accuracy of calculations (see later for an example).   $\epsilon = 2^{(1-d)}$ where $d$ is the number of bits in the fraction.
	\end{itemize}
\end{frame}

\begin{frame}[fragile]
	\frametitle{Some consequences of lack of accuracy}

		For clarity, the examples here use a base 10 representation with 4 decimal places in the fraction.
	
	\begin{itemize}
	    \item <2-> Things rarely add up exactly, due to rounding errors.
	\bes
	    1/3+1/3+1/3 &=& 1 \\
	    .3333 +.3333 + .3333 &=& .9999
	\ees
	    \item <3->Order of summation matters (associative rule doesn't work).  E.g.:
	\bes
	        1.000 + .0001 + \{\mbox{9,998 more .0001s}\} + .0001 = 1.000\\
	        .0001 + \{\mbox{9,998 more .0001s}\} + .0001 + 1.000 = 2.000
	\ees
	    \item <4->When adding $n$ values to $x$, the maximum error $<\frac{n  \epsilon x}{2}$.
	\end{itemize}
\end{frame}
\begin{frame}[fragile]
	\mode<presentation>{\frametitle{Some consequences of lack of accuracy}}
	\begin{itemize}
	
	    \item Mixed signs can cause problems -- `catastrophic cancellation': subtraction of two nearly equal numbers eliminates the most significant digits, and any small rounding errors that were previously hidden end up becoming important.
	    \begin{itemize}
	        \item <2->Poor example: $.1000 \times 10^0 - .9999 \times 10^{-1} = .1000 \times 10^{-3}$ instead of $.1000 \times 10^{-4}$ -- so the answer is wrong in what is now the most significant digit.
	        \item <3->Better example: calculating the sample variance using the usual $s^2=\frac{\sum{\left(x^2\right)}}{n} - \left(\frac{\sum{x}}{n}\right)^2$ -- can get -ve estimates!  If you want to see, try it for the sample $\{356, 357, 358, 359, 360\}$ keeping 4 decimal places in the fraction.
	
	    \end{itemize}
	\end{itemize}

\end{frame}

\begin{frame}[fragile]
	\frametitle{Floating point numbers in R}

	\begin{itemize}
    \item <1->In R, the floating point data type is called \texttt{numeric}.
		\item <2->It is the default for numbers in R.
\begin{semiverbatim}\mode<presentation>{\footnotesize}
> x <- 3
> class(x)
[1] "numeric"
\end{semiverbatim}
    \item <3->\texttt{numeric} values are represented using an IEEE standard for double precision.  64 bits in total, 1 bit for the sign, 11 bits for the exponent, 53 for the the fraction (uses a trick to gain 1 extra bit).
	\end{itemize}
\end{frame}
\begin{frame}[fragile]
	\mode<presentation>{\frametitle{Floating point numbers in R}}
	\begin{itemize}
    \item So,
    \begin{itemize}
        \item <2-> the largest possible number is\footnote{it is slightly higher, due to some tricks}
         $+.1111\ldots_{\mbox{two}} \times 2^{+1023} = 8.98\times 10^{+307}$
\begin{semiverbatim}
> 1 * 2 ^ 1023
[1] 8.988466e+307
> 2 * 2 ^ 1023
[1] Inf\end{semiverbatim}
        \item <3-> accuracy: machine epsilon is\footnote{again, it's slightly lower, due to some tricks} $\epsilon = 2^{(1-53)} = 2.20\times 10^{-16}$
\begin{semiverbatim}
> 1 + 2 ^ (1 - 53) == 1
[1] FALSE
> 1 + (0.5 * 2 ^ (1 - 53)) == 1
[1] TRUE\end{semiverbatim}
    \end{itemize}
	\end{itemize}
\end{frame}
\begin{frame}[fragile]
	\mode<presentation>{\frametitle{Floating point numbers in R}}
  \begin{itemize}
    \item <1->The standard dictates how some arithmetic operations should be done, and defines special values +Inf and -Inf (for overflows), and NaN (not a number).
\begin{semiverbatim}\mode<presentation>{\footnotesize}
> 1 / 0
[1] Inf
> 0 / 0
[1] NaN
> sqrt(-1)
[1] NaN
Warning message:
In sqrt(-1) : NaNs produced
> Inf / Inf
[1] NaN
\end{semiverbatim}
    \item <2->Underflows are set to 0, and no warning is generated.
\begin{semiverbatim}\mode<presentation>{\footnotesize}
> exp(-1000)
[1] 0
\end{semiverbatim}

\end{itemize}

\end{frame}


\begin{frame}[fragile]
	\frametitle{Living with floating point limitations}

	\begin{itemize}
	  \item Re-work expressions and algorithms to
    \begin{itemize}
      \item<2-> Avoid generating very large numbers that might overflow.  Underflow is generally better than overflow.  E.g. Rewrite $\frac{e^x}{1+e^x}$ as $\frac{1}{1+e^{-x}}$
	\begin{semiverbatim}\mode<presentation>{\footnotesize}
	> x <- 1000
	> exp(x) / (1 + exp(x))
	[1] NaN
	> 1 / (1 + exp(-x))
	[1] 1\end{semiverbatim}
	   \item <3->Minimize large differences in scale in intermediate calculations, so as to preserve accuracy.  For example, instead of taking the product of a large number of likelihoods, take the sum of the log likelihoods and then exponentiate.
	 \end{itemize}
  \end{itemize}
\end{frame}
\begin{frame}[fragile]
	\mode<presentation>{\frametitle{Living with floating point limitations}}
  \begin{itemize}	
     \item Normalize\footnote{Note: different meaning here from normalizing the fractional part of a floating point number} -- scale values so they are centred near 0, and perhaps divide by some suitably large value so they don't overflow.  Allows the maximum range of values.
	    \item <2->Change exact tests to `good enough` tests.  From\\
	\texttt{if(x == y) then...}\\
	to\\
	\texttt{if(((x - e) <= y) \& (x + e >= y)) then ...}\\
	or, better (scale independent)\\
	\texttt{if(abs(x - y) <= abs(y) * e) then ...}

  \end{itemize}
\end{frame}
\begin{frame}[fragile]
	\mode<presentation>{\frametitle{Living with floating point limitations}}
  \begin{itemize}	

	    \item Be careful about the order of operations - e.g., consider sorting numbers before adding (there are even smarter tricks here).
	
	    \item <2-> If you're desperate, use double-precision (at least for intermediate calculations) -- but it uses twice the memory and can use 3-4 times the time. (Although this does depend on the size of the floats relative to computer word size, so best to try it and see.)\footnote{Doesn't apply to R, which always uses double precision.  If you want even more accuracy, you can use the \texttt{Rmpfr} package.}
	\end{itemize}

\end{frame}
\begin{frame}[fragile]
	\mode<presentation>{\frametitle{Living with floating point limitations}}
	
	Some comments:
	\begin{itemize}
	    \item Note that many of these show a trade-off between accuracy and speed.
	    \item <2->So, first ask yourself if it's going to be a problem, before spending lots of time fixing a problem that isn't important!
	\end{itemize}
	
\end{frame}

\begin{frame}[fragile]
	\frametitle{Summary: floating point representations}

	\begin{itemize}
	    \item Advantages:
	    \begin{itemize}
	        \item Wide applicability (all numbers on the real line)
	        \item Wider range than fixed point
	    \end{itemize}
	    \item <2->Disadvantages:
	    \begin{itemize}
	        \item Results are not exact -- need to be careful
	        \item Slow (compared with fixed point)
	    \end{itemize}
	\end{itemize}
\end{frame}

\begin{frame}[fragile]
	\frametitle{Motivating example \texttt{R}edux}

What should we do instead of this?

\begin{verbatim} 
if (x == 0.15) {
  cat("Len is a great guy!\n") 
} else {
  cat("Len is a loser!\n")
}
\end{verbatim}
\end{frame}

\mode<presentation>
%This frame not shown in the handout!
\begin{frame}[fragile]
  \frametitle{Motivating example \texttt{R}edux}

Good enough test, such as:

\begin{verbatim} 
e <- 1E-15
if (abs(x - 0.15) < 0.15 * e) {
  cat("Len is a great guy!\n") 
} else {
  cat("Len is a loser!\n")
}
\end{verbatim}
\end{frame}

\mode*
%Back to showing things in the handout
\subsection{Character storage}

\begin{frame}[label=current]
	\frametitle{Character storage\footnote{OK - I know characters aren't numbers but I thought I'd include this stuff anyway}}

	\begin{itemize}
	    \item Characters are stored as binary numbers on the computer.  The correspondence between the stored numbers and the displayed symbols is determined by a character set.
	\end{itemize}
\end{frame}
\begin{frame}[label=current]
	\mode<presentation>{\frametitle{Character storage}}
	\begin{itemize}
	    \item Example character sets:
	    \begin{itemize}
	        \item ASCII (American Standard Code for Information Exchange).  A 7-bit system, so there are 128 unique characters.  0-31 are non-printing control characters (e.g. 27 is ESCape), the other 96 are alphanumeric characters and symbols (e.g. 63 is ?, 77 is M).
	        \item Extended ASCII. A set of 8-bit systems - the extra 128 characters are used for European accents, etc.  There are many regional variants.  Used in old versions of windows, and (I think) in R.
	        \item UNICODE.  A 16-bit system, so there are 65,536 characters.  Designed to include extended-character languages like Chinese and Japanese - but not really sufficient for that.  Used in MS Windows (2000 and later).
	    \end{itemize}
	\end{itemize}
\end{frame}
\begin{frame}[label=current]
	\mode<presentation>{\frametitle{Character storage}}
	\begin{itemize}
	    \item Practical note: there can be hassles switching between operating systems or locales within operating systems. E.g., Unix (LF) to Windows (CR+LF).
	\end{itemize}
\end{frame}

\subsection*{Practice questions}

\begin{frame}[fragile,label=current]
	\mode<presentation>{\frametitle{Practice questions}}
	\mode<presentation>{\footnotesize}
\bn
    \item Write out -15-3=-18 in binary using an 8-bit signed integer system
%  -15 10001111
%-  +3 00000011
%= -18 10010010		
    \item What is the largest positive number that can be represented in this system?
%01111111 = 127
    \item Why is it very fast to multiply numbers represented as integers by 2?  What manipulation is required to the bits making up the number to achieve this operation?
%Just need to shift the bits to the left (apart from the sign bit and the most significant digit -- if the latter is already 1 then you get an overflow), and put a 0 in the least significant digit bit
    \item What is the machine epsilon on a 32-bit floating point system where 1 bit is for the sign, 16 bits for the exponent and 16 bits\footnote{using the same trick as in the IEEE standard to get one extra bit for free; note that this is not a very sensible allocation as it gives far too much of the space to the exponent} for the fraction?
% epsilon = 2^(1-d)=2^(-15)
    \item In this system, what would be the result of computing $2* 2^{-17} + 1$?
%2x2^-17 = 1x2^-16 -- below the machine epsilon so the result would be 1
    \item What do you get if you compute $512 * 0.25$ in the above floating point system, and then convert it into the above signed integer system?\footnote{A calculation like this once cost the EU space program about \$500 million: \href{https://www.ima.umn.edu/~arnold/disasters/ariane.html}{https://www.ima.umn.edu/$\sim$arnold/disasters/ariane.html}}
%512*0.25 is easily evaluated exactly in the floating point system, giving 128.  This is larger than the largest integer, however, so you'd get an overflow
 \en
\end{frame}

\end{document}