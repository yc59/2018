\documentclass[11pt,a4paper]{article}
\usepackage[colorlinks,
pdfauthor="Len Thomas",
pdftitle="MT4113 Lecture notes"]{hyperref}
\usepackage[a4paper,margin=2cm,footskip=.5cm]{geometry}
\usepackage[english]{babel}
\usepackage[latin1]{inputenc}
\usepackage[T1]{fontenc}
\usepackage{epsf,graphics,graphicx,fancyhdr,color,amsmath,url,enumerate,alltt}

\begin{document}

\section*{MT4113 Lecture 4 Computer arithmetic\\Solutions to practice questions} 

\begin{enumerate}
    \item Write out -15-3=-18 in binary using an 8-bit signed integer system
\begin{verbatim}  -15 10001111
-  +3 00000011
= -18 10010010
\end{verbatim}		
    \item What is the largest positive number that can be represented in this system?\\ \\
\texttt{01111111 = 127}
    \item Why is it very fast to multiply numbers represented as integers by 2?  What manipulation is required to the bits making up the number to achieve this operation?\\

\texttt{The computer just needs to shift the bits to the left (apart from the sign bit and the most significant digit -- if the latter is already 1 then you get an overflow), and put a 0 in the least significant digit bit}

    \item What is the machine epsilon on a 32-bit floating point system where 1 bit is for the sign, 16 bits for the exponent and 16 bits\footnote{using the same trick as in the IEEE standard to get one extra bit for free; note that this is not a very sensible allocation as it gives far too much of the space to the exponent} for the fraction?\\
		
\texttt{epsilon = $2^{(1-d)}=2^{(-15)}$}

    \item In this system, what would be the result of computing $2* 2^{-17} + 1$?\\
		
\texttt{$2\times2^{-17} = 1\times2{^-16}$.  This is below the machine epsilon so the result would be 1.}

    \item What do you get if you compute $512 * 0.25$ in the above floating point system, and then convert it into the above signed integer system?\footnote{A calculation like this once cost the EU space program about \$500 million: \href{https://www.ima.umn.edu/~arnold/disasters/ariane.html}{https://www.ima.umn.edu/$\sim$arnold/disasters/ariane.html}}\\
		
\texttt{$512 \times 0.25$ is easily evaluated exactly in the floating point system, giving 128.  This is larger than the largest integer, however, so you'd get an overflow}

\end{enumerate}

\end{document}