\documentclass[ignorenonframetext,]{beamer}
\setbeamertemplate{caption}[numbered]
\setbeamertemplate{caption label separator}{: }
\setbeamercolor{caption name}{fg=normal text.fg}
\beamertemplatenavigationsymbolsempty
\usepackage{lmodern}
\usepackage{amssymb,amsmath}
\usepackage{ifxetex,ifluatex}
\usepackage{fixltx2e} % provides \textsubscript
\ifnum 0\ifxetex 1\fi\ifluatex 1\fi=0 % if pdftex
  \usepackage[T1]{fontenc}
  \usepackage[utf8]{inputenc}
\else % if luatex or xelatex
  \ifxetex
    \usepackage{mathspec}
  \else
    \usepackage{fontspec}
  \fi
  \defaultfontfeatures{Ligatures=TeX,Scale=MatchLowercase}
\fi
\usetheme[]{EastLansing}
\usecolortheme{crane}
\usefonttheme{structurebold}
% use upquote if available, for straight quotes in verbatim environments
\IfFileExists{upquote.sty}{\usepackage{upquote}}{}
% use microtype if available
\IfFileExists{microtype.sty}{%
\usepackage{microtype}
\UseMicrotypeSet[protrusion]{basicmath} % disable protrusion for tt fonts
}{}
\newif\ifbibliography
\hypersetup{
            pdftitle={MT4113, Computing in Statistics},
            pdfauthor={Lecture 1 - Hardware, software and algorithms},
            pdfborder={0 0 0},
            breaklinks=true}
\urlstyle{same}  % don't use monospace font for urls

% Prevent slide breaks in the middle of a paragraph:
\widowpenalties 1 10000
\raggedbottom

\AtBeginPart{
  \let\insertpartnumber\relax
  \let\partname\relax
  \frame{\partpage}
}
\AtBeginSection{
  \ifbibliography
  \else
    \let\insertsectionnumber\relax
    \let\sectionname\relax
    \frame{\sectionpage}
  \fi
}
\AtBeginSubsection{
  \let\insertsubsectionnumber\relax
  \let\subsectionname\relax
  \frame{\subsectionpage}
}

\setlength{\parindent}{0pt}
\setlength{\parskip}{6pt plus 2pt minus 1pt}
\setlength{\emergencystretch}{3em}  % prevent overfull lines
\providecommand{\tightlist}{%
  \setlength{\itemsep}{0pt}\setlength{\parskip}{0pt}}
\setcounter{secnumdepth}{0}

\title{MT4113, Computing in Statistics}
\author{Lecture 1 - Hardware, software and algorithms}
\date{17 September 2018}

\begin{document}
\frame{\titlepage}

\begin{frame}
\tableofcontents[hideallsubsections]
\end{frame}

\begin{frame}

\end{frame}

\section{Computers (from a statistician's
persepctive)}\label{computers-from-a-statisticians-persepctive}

\begin{frame}{What is a computer?}

\begin{itemize}[<+->]
\tightlist
\item
  Approximate definition:
\end{itemize}

\begin{quote}
A device that accepts information (in the form of digitized data) and
manipulates it according to a sequence of instructions to produce
output.
\end{quote}

\begin{itemize}[<+->]
\tightlist
\item
  We can divide the device into two parts:

  \begin{itemize}[<+->]
  \tightlist
  \item
    Hardware - physical computer equipment
  \item
    Software - set of instructions that operate the hardware
    \textasciitilde{} computer program
  \end{itemize}
\end{itemize}

\end{frame}

\begin{frame}{Computer architecture}

Hardware

\begin{itemize}[<+->]
\tightlist
\item
  Central Processing Unit (CPU) -- controls the computer and executes
  instructions\\
  (N.B. Many computers also contain graphics processing units (GPU's),
  which can be used for numerical computation.)
\item
  Memory -- fast storage

  \begin{itemize}[<+->]
  \tightlist
  \item
    Registers -- within the CPU, superfast
  \item
    Random Access Memory (RAM) -- slower
  \end{itemize}
\item
  Mass storage -- e.g., hard disk drives -- (traditionally) much slower
\item
  Input/Output -- keyboard, screen, printer, etc.
\end{itemize}

\end{frame}

\begin{frame}

Software

\begin{itemize}[<+->]
\tightlist
\item
  System software -- e.g., Microsoft Windows, Linux
\item
  Application software -- e.g., Microsoft Word, SAS, R
\item
  Programming software -- e.g., Microsoft Visual C, GNU C, R

  \begin{itemize}[<+->]
  \tightlist
  \item
    Implementations of programming languages -- C, S, etc.
  \end{itemize}
\end{itemize}

\end{frame}

\begin{frame}{Programming languages}

Classified according to:

\begin{itemize}[<+->]
\tightlist
\item
  Level of abstraction

  \begin{itemize}[<+->]
  \tightlist
  \item
    low level = close to the specific computer hardware - black-belt
    stuff
  \item
    high level = far from the hardware (so can run on many different
    systems) and closer to natural language - can focus on tasks to be
    achieved
  \end{itemize}
\item
  Generality: higher level languages tend to be more specialized in
  application
\item
  Speed/Efficiency: lower level is generally faster (although not always
  true)
\item
  Generations: 1st-5th
\end{itemize}

\end{frame}

\begin{frame}[fragile]{Programming language generations}

\begin{itemize}[<+->]
\tightlist
\item
  1st generation (1GL): machine language - CPU-specific set of
  instructions. E.g. \texttt{10110000\ 01100001} on a Pentium-type CPU
  means move the value 97 into a certain register.

  \begin{itemize}[<+->]
  \tightlist
  \item
    Relatively few different instructions (\(\approx100\) in an Intel
    chip) and many of these are required to achieve anything useful.
  \end{itemize}
\item
  2nd generation (2GL): assembly language.

  \begin{itemize}[<+->]
  \tightlist
  \item
    A human-readable form of machine language. E.g., the above in
    assembly is \texttt{mov\ al,\ 061h}
  \item
    \texttt{al} is the name of the register and \texttt{61h} is 61 in
    hexadecimal, which is 97 in decimal
  \end{itemize}
\end{itemize}

\end{frame}

\begin{frame}[fragile]

\begin{itemize}[<+->]
\tightlist
\item
  3rd generation (3GL): more human-friendly, CPU-independent language
  with variables, data and code structures.

  \begin{itemize}[<+->]
  \item
    E.g.\\
    FORTRAN

\begin{verbatim}
IF(X.EQ.12) LET A = B
\end{verbatim}

    BASIC

\begin{verbatim}
if (x=12) then a = b
\end{verbatim}

    Pascal

\begin{verbatim}
if (x=12) then a := b;
\end{verbatim}

    C

\begin{verbatim}
if (x==12) a = b
\end{verbatim}
  \item
    Also \emph{object-oriented} languages such as C++ and Java.
  \item
    Most programs you use (e.g., Windows, Unix, R, etc.) are written in
    a 3GL.
  \end{itemize}
\end{itemize}

\end{frame}

\begin{frame}

\begin{itemize}[<+->]
\tightlist
\item
  4th generation (4GL): language designed with a specific application in
  mind -- lots of built-in capabilities for that application.

  \begin{itemize}[<+->]
  \tightlist
  \item
    database query languages (SQL), graphical user interface (GUI)
    creators (Visual Basic), mathematics languages (Mathematica, Maple),
    statistics languages (R, SAS).
  \end{itemize}
\item
  5th generation (5GL): language based around solving problems, given
  the problem specification. User does not need to explicitly write the
  algorithm for solving the problem.

  \begin{itemize}[<+->]
  \tightlist
  \item
    Prolog. Used is in artificial intelligence studies.
  \end{itemize}
\end{itemize}

\end{frame}

\begin{frame}{Compiled vs Interpreted}

\begin{itemize}[<+->]
\tightlist
\item
  All languages must be turned into machine code before they can be
  executed by a computer
\item
  Two approaches:

  \begin{itemize}[<+->]
  \tightlist
  \item
    Interpret: turn each line into machine code as it is entered using
    an `interpreter' and run it straight away.
  \item
    Compile: once the code is written and saved into a file, turn it all
    into machine code in one go using a `compiler'. Then, it can be run.
  \end{itemize}
\item
  Pros and cons:

  \begin{itemize}[<+->]
  \tightlist
  \item
    Interpreted code provides instant feedback -- good for short, run
    once jobs. Tend to be used by 4GLs.
  \item
    Compiled code runs faster (compiler can \emph{optimize}) -- good for
    jobs that will be run many times. Tend to be used by 3GLs.
  \end{itemize}
\end{itemize}

\end{frame}

\begin{frame}[fragile]{(Trivial) example of optimization}

\begin{itemize}[<+->]
\item
  Slower:

\begin{verbatim}
for i = 1 to 10
  j[i] = log(k) + i
next
\end{verbatim}
\item
  An optimizing compiler could recognize the inefficiency and
  automatically turn it into something like the following while
  compiling.
\item
  Faster:

\begin{verbatim}
l = log(k)
for i = 1 to 10
  j[i] = l + i
next
\end{verbatim}
\end{itemize}

\end{frame}

\begin{frame}{What programming language/package should we use for
statistics?}

\begin{itemize}[<+->]
\tightlist
\item
  For simple \emph{do once} stuff (by the way, \emph{do once} is a
  myth), we can use our favourite stats package, perhaps via a GUI

  \begin{itemize}[<+->]
  \tightlist
  \item
    but there is no \emph{reproducible trail} of how the analysis was
    done
  \end{itemize}
\item
  For more complex stuff, but still only \emph{do once} or \emph{do a
  few times}, use a statistical 4GL within an interpreter
\item
  For production software, or where efficiency is important

  \begin{itemize}[<+->]
  \tightlist
  \item
    use a 3GL and compile the code, or
  \item
    prototype in a 4GL and then re-write the bits that are slow in a 3GL
    which you call from the 4GL
  \end{itemize}
\end{itemize}

\end{frame}

\section{Software for statistics}\label{software-for-statistics}

Most of the time you have no choice!

\begin{itemize}[<+->]
\tightlist
\item
  Use what the client uses/wants
\item
  Your company only has a license for one piece of software, or your IT
  department will only support one package
\item
  You are taking MT4113 and we are only teaching you R
\end{itemize}

\begin{frame}{If you do have a choice:}

\begin{itemize}[<+->]
\tightlist
\item
  If you have a zero budget, use R -- it's free!
\item
  If you only use the package rarely and want to do something quite
  standard, choose something easy to use (point and click) SPSS, etc.
\item
  For enormous datasets, use SAS (or maybe Microsoft R Open, or R with
  Hadoop)
\item
  If you are doing research in statistics, start with R (or maybe
  MATLAB/Mathematica)
\end{itemize}

\end{frame}

\begin{frame}

\begin{itemize}[<+->]
\tightlist
\item
  For computer-intensive work, or something that you wish to automate

  \begin{itemize}[<+->]
  \tightlist
  \item
    Prototype with an easy-to-use but extensible package (R)
  \item
    Find out which bits are slow and port them to a 3GL
  \item
    \ldots{} or re-write the whole thing into a 3GL
  \item
    Consider what kind of user interface you need
  \end{itemize}
\item
  For specialized applications, look at the specialized packages
\end{itemize}

\end{frame}

\begin{frame}{R}

\begin{itemize}[<+->]
\tightlist
\item
  a language and environment for statistical computing and graphics
\item
  open source, free
\item
  widely used by academic statisticians
\item
  based on the S 4GL
\item
  extremely extensible
\item
  rapidly developing and maturing
\end{itemize}

\end{frame}

\begin{frame}[fragile]

\begin{itemize}[<+->]
\tightlist
\item
  Current version is 3.5.1
\item
  Widely used in academia, making its mark elsewhere
\item
  Core development team of approx 20 members, with many hundreds
  contributing extensions ( packages )
\item
  Supported by the R Foundation and (commercial) R Consortium.
\item
  \texttt{useR!} conferences, etc.
\end{itemize}

\end{frame}

\begin{frame}{Why use R?}

\begin{itemize}[<+->]
\tightlist
\item
  Pros:
\item
  Contains cutting-edge methods (usually as add-on packages)

  \begin{itemize}[<+->]
  \tightlist
  \item
    Highly extensible
  \item
    Cannot beat the price
  \end{itemize}
\item
  Cons:

  \begin{itemize}[<+->]
  \tightlist
  \item
    Steep learning curve
  \item
    Less well supported than a commercial package?
  \item
    Greater tendency to ignore backwards compatibility?
  \end{itemize}
\end{itemize}

\end{frame}

\begin{frame}{Learning resources for R}

\begin{itemize}[<+->]
\tightlist
\item
  Introduction to Stats with R books (see reading list)
\item
  R project home page \url{http://www.r-project.org/}

  \begin{itemize}[<+->]
  \tightlist
  \item
    Up-to-date list of books
  \end{itemize}
\item
  Other books see additional reading list (on Moodle)
\item
  R bloggers - a collection of hundreds of blogs on R
  \url{http://www.r-bloggers.com/}
\end{itemize}

\end{frame}

\begin{frame}{Evolution of an R programmer}

\begin{itemize}[<+->]
\tightlist
\item
  Phase 1

  \begin{itemize}[<+->]
  \tightlist
  \item
    R as calculator
  \end{itemize}
\item
  Phase 2

  \begin{itemize}[<+->]
  \tightlist
  \item
    the \emph{script}
  \item
    10-15 lines of code strung together to perform a task
  \item
    might have a loop, maybe even an if-statement
  \item
    one or two comments
  \item
    not intended for use (or viewing) by others
  \end{itemize}
\item
  Phase 3

  \begin{itemize}[<+->]
  \tightlist
  \item
    modular programming with functions
  \item
    undertaking a sufficiently complex analysis that organisation
    becomes critical
  \item
    description of input and output are critical to ensure reusability
  \end{itemize}
\item
  When performing functional programming,

  \begin{itemize}[<+->]
  \tightlist
  \item
    there are elements of the R language not previously used that become
    useful
  \end{itemize}
\end{itemize}

\end{frame}

\section{Algorithms and computer
code}\label{algorithms-and-computer-code}

Note to self - expand again?

\begin{itemize}[<+->]
\tightlist
\item
  Algorithm: an \emph{ordered} sequence of \emph{unambiguous} and
  well-defined instructions for \emph{performing some task} and
  \emph{halting} in finite time
\item
  Important features

  \begin{itemize}[<+->]
  \tightlist
  \item
    an ordered sequence
  \item
    unambiguous and well defined instructions each instruction is clear,
    do-able, and can be done without difficulty
  \item
    performs some task algorithm needs to be complete, with nothing left
    out
  \item
    halts in finite time i.e., the algorithm needs to terminate
  \end{itemize}
\end{itemize}

\section{Description of Assignment 1}\label{description-of-assignment-1}

Goes here!

\end{document}
