\documentclass[ignorenonframetext,]{beamer}
\setbeamertemplate{caption}[numbered]
\setbeamertemplate{caption label separator}{: }
\setbeamercolor{caption name}{fg=normal text.fg}
\beamertemplatenavigationsymbolsempty
\usepackage{lmodern}
\usepackage{amssymb,amsmath}
\usepackage{ifxetex,ifluatex}
\usepackage{fixltx2e} % provides \textsubscript
\ifnum 0\ifxetex 1\fi\ifluatex 1\fi=0 % if pdftex
  \usepackage[T1]{fontenc}
  \usepackage[utf8]{inputenc}
\else % if luatex or xelatex
  \ifxetex
    \usepackage{mathspec}
  \else
    \usepackage{fontspec}
  \fi
  \defaultfontfeatures{Ligatures=TeX,Scale=MatchLowercase}
\fi
\usetheme[]{EastLansing}
\usecolortheme{crane}
\usefonttheme{structurebold}
% use upquote if available, for straight quotes in verbatim environments
\IfFileExists{upquote.sty}{\usepackage{upquote}}{}
% use microtype if available
\IfFileExists{microtype.sty}{%
\usepackage{microtype}
\UseMicrotypeSet[protrusion]{basicmath} % disable protrusion for tt fonts
}{}
\newif\ifbibliography
\hypersetup{
            pdftitle={MT4113, Computing in Statistics},
            pdfauthor={Lecture 0 - Introduction to Module},
            pdfborder={0 0 0},
            breaklinks=true}
\urlstyle{same}  % don't use monospace font for urls

% Prevent slide breaks in the middle of a paragraph:
\widowpenalties 1 10000
\raggedbottom

\AtBeginPart{
  \let\insertpartnumber\relax
  \let\partname\relax
  \frame{\partpage}
}
\AtBeginSection{
  \ifbibliography
  \else
    \let\insertsectionnumber\relax
    \let\sectionname\relax
    \frame{\sectionpage}
  \fi
}
\AtBeginSubsection{
  \let\insertsubsectionnumber\relax
  \let\subsectionname\relax
  \frame{\subsectionpage}
}

\setlength{\parindent}{0pt}
\setlength{\parskip}{6pt plus 2pt minus 1pt}
\setlength{\emergencystretch}{3em}  % prevent overfull lines
\providecommand{\tightlist}{%
  \setlength{\itemsep}{0pt}\setlength{\parskip}{0pt}}
\setcounter{secnumdepth}{0}

\title{MT4113, Computing in Statistics}
\author{Lecture 0 - Introduction to Module}
\date{17 September 2018}

\begin{document}
\frame{\titlepage}

\begin{frame}{Welcome to MT4113 Computing in Statistics!}

Module instructors:

\begin{itemize}[<+->]
\tightlist
\item
  Len Thomas
  \href{mailto:len.thomas@st-andrews.ac.uk}{\nolinkurl{len.thomas@st-andrews.ac.uk}}
\item
  Eiren Jacobson
  \href{mailto:ej45@st-andrews.ac.uk}{\nolinkurl{ej45@st-andrews.ac.uk}}
\end{itemize}

Module tutor:

\begin{itemize}[<+->]
\tightlist
\item
  Fanny Empacher
  \href{mailto:fe21@st-andrews.ac.uk}{\nolinkurl{fe21@st-andrews.ac.uk}}
\end{itemize}

\end{frame}

\begin{frame}{Why am I taking this module?}

This module will:

\begin{itemize}[<+->]
\tightlist
\item
  Help me get a job (in industry or research)

  \begin{itemize}[<+->]
  \tightlist
  \item
    Computers are essential tools in both pure and applied statistics,
    and most other walks of life
  \item
    Many job applications require experience with given computer
    packages
  \item
    Here, we'll be using one of the dominant packages in statistical
    computing: R
  \end{itemize}
\end{itemize}

\end{frame}

\begin{frame}

\begin{itemize}[<+->]
\tightlist
\item
  Help me keep that job

  \begin{itemize}[<+->]
  \tightlist
  \item
    Sooner or later, we need to automate a set of tasks, or extend
    software to perform new tasks
  \item
    For this to go smoothly, we need to know how to write efficient,
    re-useable programs that work
  \item
    Good programming practice will be a major theme of this module -- a
    very transferrable skill
  \end{itemize}
\end{itemize}

\end{frame}

\begin{frame}{Why am I taking this module? III}

This module will:

\begin{itemize}[<+->]
\tightlist
\item
  Help me out of trouble

  \begin{itemize}[<+->]
  \tightlist
  \item
    Many statistical methods make strong assumptions about the
    underlying distribution of the data.
  \item
    By contrast, we will (briefly) cover some some simple, robust,
    computer-intensive methods that are very powerful yet make only weak
    assumptions about the data.
  \item
    We will also cover methods for optimizing functions -- useful for
    things like maximizing likelihoods.
  \end{itemize}
\end{itemize}

\end{frame}

\begin{frame}{Why am I taking this module? IV}

This module will:

\begin{itemize}[<+->]
\tightlist
\item
  Help me with other modules

  \begin{itemize}[<+->]
  \tightlist
  \item
    R and general programming skills will be useful in other modules and
    project work.
  \end{itemize}
\end{itemize}

\end{frame}

\begin{frame}{Why am I taking this module? V}

\begin{itemize}[<+->]
\tightlist
\item
  Err\ldots{} my course requires me to take a computing module and this
  looked the least worst option!
\end{itemize}

\end{frame}

\begin{frame}{Approximate Syllabus}

\begin{itemize}[<+->]
\tightlist
\item
  Fundamental programming principles (Len \& Eiren), Weeks 1-3
\item
  Computer-intensive statistics (Len), Weeks 4-5
\item
  Optimization methods (Eiren), Weeks 7-8
\item
  Code performance, reprodcibility and graphics (Eren \& Len), Weeks
  9-10
\end{itemize}

\end{frame}

\begin{frame}{Timetable}

\begin{itemize}[<+->]
\tightlist
\item
  Format: Mix of lectures (Monday LTC, Wednesday LTA, 12-1pm) and
  computer practicals (Friday Comp. classroom, 12-2pm)
\item
  Drop-in help sessions: Likely to be Wednesday 1-2pm LTC, with Fanny
\item
  Assessment: 60\% in-semester assessment; 40\% final exam

  \begin{itemize}[<+->]
  \tightlist
  \item
    1 smaller programming assignment, worth 10\%.

    \begin{itemize}[<+->]
    \tightlist
    \item
      Peer review exercise, for additional feedback.
    \end{itemize}
  \item
    1 group programming project, worth 20\%.

    \begin{itemize}[<+->]
    \tightlist
    \item
      1 larger programming project, worth 30\%.
    \end{itemize}
  \end{itemize}
\item
  Module web site: on Moodle

  \begin{itemize}[<+->]
  \tightlist
  \item
    Includes fora (news and help), lectures, projects, additional
    material
  \item
    The Moodle site is the module handbook.
  \end{itemize}
\end{itemize}

\end{frame}

\begin{frame}{Deadlines}

\begin{itemize}[<+->]
\tightlist
\item
  Published on the module web site
\item
  University late coursework policy is linked on the web site
\item
  If you have a registered request for flexible deadlines, please
  discuss this with me or Eiren at the end of this class
\end{itemize}

\end{frame}

\begin{frame}[fragile]{Attendence Monitoring}

New University-level policy

\begin{verbatim}
As with all 3000-5000 Level taught modules in the School of
Mathematics and Statistics attendance will be monitored at 
various points over the semester. This is to comply with UKVI 
regulations and to monitor student wellbeing.  
For MT4113 attendance will be taken at the practicals of 
Friday 12th October and Friday 16th November.
\end{verbatim}

\end{frame}

\begin{frame}[fragile]{Notice}

\begin{verbatim}
    This module requires considerable time investment to 
    reach your potential (nominal work load for all 15 
    credit modules is 150 hours, including class time).
\end{verbatim}

\end{frame}

\end{document}
